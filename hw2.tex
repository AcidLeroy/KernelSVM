%{
\documentclass[11pt, twoside]{article}   	% use "amsart" instead of "article" for AMSLaTeX format
\usepackage{geometry}                		% See geometry.pdf to learn the layout options. There are lots.
\geometry{letterpaper}                   		% ... or a4paper or a5paper or ... 
%\geometry{landscape}                		% Activate for for rotated page geometry
\usepackage[parfill]{parskip}    		% Activate to begin paragraphs with an empty line rather than an indent
\usepackage{graphicx}				% Use pdf, png, jpg, or eps§ with pdflatex; use eps in DVI mode
								% TeX will automatically convert eps --> pdf in pdflatex	
\usepackage{amssymb}
\usepackage{color}
\usepackage{matlab-prettifier}
\usepackage{verbatim}
\usepackage{fancyvrb}
\usepackage{multicol}
\usepackage{amsmath}
\usepackage{cite}
\usepackage{bm}
\usepackage[section]{placeins}
\lstset{style=Matlab-editor,basicstyle=\ttfamily}

\sloppy
\definecolor{lightgray}{gray}{0.5}
\newenvironment{matlab}{\comment}{\endcomment}
\newenvironment{matlabv}{\lstlisting}{\endlstlisting}	


\title{Homework \# 2 \\ Kernels}
\author{Cody W. Eilar}
%\date{}							% Activate to display a given date or no date

\begin{document}
\maketitle


\begin{matlab}
%}
close all; 
clear all;

fid = fopen('output_dir/computer.tex', 'w'); 
fprintf(fid, computer); 
fclose(fid); 

fid = fopen('output_dir/matlabver.tex', 'w'); 
a = ver('matlab'); 
fprintf(fid, [a.Name ' version ' a.Version]); 
fclose(fid); 
%{
\end{matlab}
\section{Introduction} 
By themselves, support vector machines are useful but only for the linearly separable case. There are many cases in which no single line can easily shatter the points 
using a hyperplane. This is where the \textit{kernel trick} comes in handy. The kernel trick consists of transforming the input space into a higher dimensional space in which a "linear"
hyperplane can be drawn. This allows the \textit{xor} problem, or any other seemingly nonlinearly separable case to be solved. In this paper, we will investigate how to solve the \textit{xor} 
and the \textit{fibonacci spiral} problems by using kernels and support vector machines. 

\section{Theory}
Before delving into the kernel trick, it also helpful to look at what possibilities came before the kernel trick was discovered. This was known as a \textit{Volterra filter}, and this allowed
the transformation of the input space into a polynomial function. Equation \ref{eq:volterra} illustrates an order 3 Volterra filter \cite{volterra}.


\begin{equation}
\mathbf{z}_n = [1, x[n], x[n-1], x^2[n], x^2[n-1], x[n] x[n-1], x[n]^3, ...]^T
\label{eq:volterra}
\end{equation}

From Equation \ref{eq:volterra} we can then reformulate the problem as Equation \ref{eq:reform} by stating that the function $\hat{y}[n]$ is an approximation of the actual output
given by some parameters multiplied by the input data. The problem, however, with this approach is the \textit{the curse of dimensionality}.

\begin{equation}
\hat{y}[n] = \mathbf{w}^T\mathbf{z}_n
\label{eq:reform}
\end{equation}

 In the order 3 Voltera filter, we see that in order to pass from a space of dimension 2, it requries $fact(5, 3) = 10$, but if the input space is in 5 dimensions, then we end up with 56 elements needed.
 This is thus known as the curse of dimensionality because when increase the number of dimensions, the number of elements begins to explode and is therefore not practical for computation
 for higher order spaces. Luckily, we have the \textit{kernel trick} to help solve this curse. The transformation can be generalized by $\mathbf{z}_n = \mathbf{\varphi}(\mathbf{x}_n)$ 
 which in turn allows us to rewrite the estimator function as shown in Equation \ref{eq:rewrite}. 
 
 \begin{equation}
\hat{y}[n] = \mathbf{w}^T\mathbf{\varphi}(\mathbf{x}_n) + b
\label{eq:rewrite}
\end{equation}

In order to finally solve Equation \ref{eq:rewrite} we need to choose a function for $\varphi(\cdot)$, the higher dimensional Hilbert space \cite{kernelmethods}. In 
the case of this paper, we have chosen to use the minimum mean square error (MMSE) criterion and a Gaussian Kernel with $\sigma = 1$. We know that we can choose this
as our kernel because the only requirement is that it be a proper inner product. To state it more explicitly, a kernel function is one that satisfies Mercer's Theorem \cite{kernelmethods}. The gaussian equation 
used is shown in Equation \ref{eq:gaus_basis}.

\begin{equation}
\mathbf{\varphi(x)} = exp\bigg(\frac{||\mathbf{x} - \mathbf{x}^T||^2}{\sigma^2}\bigg)
\label{eq:gaus_basis}
\end{equation}

From here, are given the optimizer which is shown in Equation \ref{eq:optimizer} 

\begin{equation}
\mathbf{\alpha} = (\mathbf{\widetilde{K}} + \gamma \mathbf{I})^{-1}\mathbf{y}
\label{eq:optimizer}
\end{equation}

Where $\mathbf{\widetilde{K}} = \mathbf{K} + \mathbf{1}$ and $\mathbf{1}$ is an $N \times N$ matrix of ones. This is a direct result
of building the bias into the Hilbert space $\mathbf{\varphi(\cdot)}$. We can show this in the following steps. 

Let 
\begin{align*}
\mathbf{\widetilde{\varphi}(x)} &= [1, \varphi_1(\mathbf{x}),  \ldots, \varphi_N(\mathbf{x}), \ldots ] \\
\langle\mathbf{\widetilde{\varphi}(x)}, \mathbf{\widetilde{w}} \rangle &= \mathbf{w}^T\mathbf{\varphi(\mathbf{x}}) + b 
\end{align*}

As a result, end up with the following definitions for $\mathbf{\widetilde{w}}$ and $\mathbf{\widetilde{\varphi}}$.
\newcommand{\newPhi}{\mathbf{\widetilde{\varphi}}}

\begin{align*}
 \mathbf{\widetilde{w}} &= 
\begin{bmatrix}
b & \mathbf{w}  \\
\end{bmatrix} \\
\widetilde{\varphi} &= 
\begin{bmatrix} 
\mathbf{1}^T \\
\mathbf{\varphi} \\
\end{bmatrix}
\end{align*}

From here, we simply need to take the inner product of $\mathbf{\widetilde{\varphi}}$ to obtain $\mathbf{K}$. 
\begin{align*}
 \widetilde{\mathbf{K} }&= \langle \newPhi(\mathbf{x}_j), \newPhi(\mathbf{x}_i) \rangle \\
 & = \newPhi^T\newPhi \\
 &= [\mathbf{1}^T, \mathbf{\varphi}] 
 \begin{bmatrix}
 \mathbf{1}^T \\
\mathbf{\varphi} \\
 \end{bmatrix} \\
 &= \mathbf{1}_{N \times N} + \mathbf{\varphi}^T\mathbf{\varphi}
\end{align*}

This then proves that in order to calculate the bias using the Hilbert space, we need to first
build the bias into the $\varphi$ variable so that we don't have to treat the equation separately. 

It is also interesting to look at why it is necessary to have $\mathbf{alpha} = (\mathbf{\widetilde{K}} - \gamma \mathbf{I})^{-1} \mathbf{y}$ 
instead of $\mathbf{\widetilde{K}}^{-1}\mathbf{y}$. The reason for this is that we cannot guarantee that $\mathbf{\widetilde{K}}$ is well posed. 
For example, if there are more points than dimensions, than there are an infinite number of $\mathbf{w}$ values from which to choose. On 
the other hand, if there are many more points than dimensions then you can be rest assured that there is noise in the generation process 
and if you attempt to choose an exact match $\mathbf{w}$ then you run the risk of over fitting the data and therefore there is a need for an 
estimation criterion \cite{kernelmethods}. The process of developing these two constraints is called \emph{regularization} or \emph{ridge regression}. 
The \emph{ridge regression} corresponds to solving the following equation: 



\begin{align*}
\underset{\mathbf{w}}{\text{min}} \mathbf{\mathcal{L}}_\lambda (\mathbf{w}, S) &= 
\underset{\mathbf{w}}{\text{min}} \lambda ||\mathbf{w}||^2 + \sum_{i=1}^{l}(y_i - g(\mathbf{x}_i))^2
\end{align*}

Here, $\lambda$ represents the relative trade-off between the norm and the loss \cite{kernelmethods}. In order to find the minimum of this equation, 
we need to take the derivative. 

\begin{align*}
\mathbf{X}^T\mathbf{X}\mathbf{w} + \lambda\mathbf{w} = (\mathbf{X}^T\mathbf{X} + \lambda \mathbf{I}_n)\mathbf{w} = \mathbf{X}^T\mathbf{y}
\end{align*}

Here, $\mathbf{X}$ is just the input space where each row is $\mathbf{x}_1, \ldots, \mathbf{x}_N$. We have, however, for this problem
cast the input space into an infinite dimensional Hilbert space, so we end up with the following: 

\begin{align*}
\mathbf{\Phi}^T\mathbf{\Phi}\mathbf{w} + \lambda\mathbf{w} = (\mathbf{\Phi}^T\mathbf{\Phi} + \lambda \mathbf{I}_n)\mathbf{w} = \mathbf{\Phi}^T\mathbf{y}
\end{align*}

Then we can solve for $\mathbf{w}$: 
\begin{align*}
\mathbf{w} = \lambda^{-1} \mathbf{\Phi}^T (\mathbf{y} - \mathbf{\Phi} \mathbf{w}) = \mathbf{\Phi}\mathbf{\alpha}
\end{align*}

This is possible because we know that $\mathbf{w}$ can be written as a linear combination of the input space. And then finally we can solve
for $\mathbf{\alpha}$.

\begin{align*}
\bm{\alpha} = \lambda^{-1}(\mathbf{y} - \mathbf{\Phi} \mathbf{w}) \\
\mathbf{y} = (\mathbf{\Phi \Phi}^T + \lambda\mathbf{I}_N)\mathbf{\alpha}
\end{align*}

From here we can see that all we have to do is a substitution from our original equation in \ref{eq:optimizer} to see that
$\lambda\bm{I}_N$ is just $\gamma\bm{I}_N$ which is used as a regularization parameter to minimize the complexity 
of the system. 

\section{Methodology and Experiments}
Now that we have introduced some of the theory behind kernels and regressors, we can now begin to look at how to 
actually implement them in software to get some experimental results. 
In the first part of this section, we want to look at how well the SVM works using
100 data points for training, and then 1000 points for testing. We will
then validate the methods using \textit{MATLABS}'s v-fold function. The
second part is to then investigate the double fibonacci spiral applying
the same techniques as mentioned above. For all the results generated here
we used \input{output_dir/computer} for the system running
\input{output_dir/matlabver}.
\subsection{XOR problem}
We begin this section by first illustrating the \textit{xor} problem. In figure 
\ref{fig:xor_plot}, we label one class in red, and the other class in blue. From 
this figure, we can see that there is no obvious way to separate the data in a
two dimensional space. But what happens if we then convert it to an N-dimensional 
space? 

\begin{matlab}
%}
num_train_points = 100; 
rng('default') 
train_seed = rng(1); 
[x_train, y_train] = generate_xor_data(num_train_points, train_seed);
xor_plot = plot_xor_2d(x_train); 
s = sprintf('Plot of XOR problem using %d points', num_train_points);
title(s)
saveas(xor_plot, 'figures/xor_plot.eps', 'epsc')
%{
\end{matlab}

\begin{figure}[h]
\centering
\includegraphics[width=4in]{figures/xor_plot.eps}
\caption{XOR plot}
\label{fig:xor_plot} 
\end{figure}

\FloatBarrier
To test this out, we must first 

\begin{matlab}
%}
gamma = 2; 
sigma = 1; 
K=kernel_matrix('rbf',x_train,x_train,sigma)+1; 
% Calculate the alphas from the dual parameters
alpha=pinv(K+gamma*eye(size(K)))*y_train;
%{
\end{matlab}

To see how well we are doing, we can plot the separating hyperplane. 
\begin{matlab}
%}
[X1,X2] = meshgrid(min(x_train(:)):.05:max(x_train(:)), ...
   min(x_train(:)):.05:max(x_train(:)));     % Costruct a grid of points for 
                                          % representation of the data.
xt=[X1(:) X2(:)]';                        % Vectorize it to use it in test 
y_=alpha'*kernel_matrix('rbf', x_train, xt, sigma)+sum(alpha);    
                                          % Test
Z=buffer(y_,size(X1,1),0);                % convert the output vector into
                              % Plot a contour over the data. 
contour(X1,X2,Z,[0 0],'LineWidth',2,'Color','Black')
title('Contour plot of the hyperplane using a radial basis kernel')
saveas(xor_plot, 'figures/xor_contour_plot.eps', 'epsc')



%{
\end{matlab}

\begin{figure}[h]
\centering
\includegraphics[width=4in]{figures/xor_contour_plot.eps}
\caption{Hyperplane seperation using a radial basis function}
\label{fig:xor_contour_plot} 
\end{figure}

To put the algorithm to the test now, we need try it on some testing data
and use cross validation to test its performance. In the following code
snippet we will generate 1000 points that we will use for testing given our
original parameters stored in $\bm{\alpha}$.

\begin{matlab}
%}
%%
kfold = 5; 
num_train_points = 1000; 
[x_train, y_train] = generate_xor_data(num_train_points, train_seed);
indicies = crossvalind('Kfold', num_train_points, kfold); 
% Classperf only works on non-negative numbers, hence the plus 1
cp = classperf(y_train+1);
for i = 1:kfold
   test = (indicies == i); train = ~test;
   K=kernel_matrix('rbf',x_train(:,train), x_train(:, train), sigma)+1; 
   alpha=pinv(K+gamma*eye(size(K)))*y_train(train);
   
   y_ = alpha'*kernel_matrix('rbf', x_train(:,train), ...
      x_train(:,test), sigma) + sum(alpha); 
   y_ = sign(y_);
   
   % Function only works on non-negative numbers, hence the plus 1
   classperf(cp, y_+1, test);
end
delete 'output_dir/classify_stats.txt'
diary 'output_dir/classify_stats.txt'
cp.display
diary off
%{
\end{matlab}

The output of running the kfold on 1000 test is detailed in the following
output from the script above: 
\color{lightgray}
\VerbatimInput{output_dir/classify_stats.txt}
\color{black}
As can be seen above, the classifier is working very well with a classification rate of nearly 98\%. 

\subsection{Fibonacci Spiral Problem}
To continue with our experiments, we want to now perform classification on
the fibonacci spiral. The fibonacci sprial is shown in Figure \ref{fig:fibonacci_spiral}. 

\begin{matlab}
%}
%%
close all;
num_samples = 100; 
[x1, x2, y] = fibonacci_data(num_samples, train_seed)
fib_fig = plot_fibonacci_spirals(x1, x2); 
title('Fibonacci Spiral')
saveas(gcf, 'figures/fibonacci_spiral.eps', 'epsc')
%{
\end{matlab}

\begin{figure}[h]
\centering
\includegraphics[width=4in]{figures/fibonacci_spiral.eps}
\caption{Plot of the Fibonacci Spiral}
\label{fig:fibonacci_spiral} 
\end{figure}


\begin{matlab}
%}

%%
[x1, x2, y_train] = fibonacci_data(num_samples, train_seed);
x_train = [x1, x2];
test_seed = rng(2); 
[x1_test, x2_test, y_test] = fibonacci_data(1000, test_seed);
x_test = [x1_test, x2_test];
sigma_vals = linspace(0.0001, 1, 20); 
gamma_vals = linspace(0.0001, 1, 20);

error_rates = zeros(numel(sigma_vals), numel(gamma_vals)); 
for sig_i = 1:numel(sigma_vals)
   for gam_i = 1:numel(gamma_vals); 
      error_rates(sig_i, gam_i) = performance(x_train, y_train, x_test, y_test, gamma_vals(gam_i), sigma_vals(sig_i)); 
   end
end

%%
figure()
imagesc(sigma_vals, gamma_vals, error_rates');
title('Visualization of Error Rates, sigma vs cost')
h = colorbar;
ylabel(h, 'ErrorRate');
xlabel('Sigma')
ylabel('Gamma'); 
saveas(gcf, 'figures/sigma_cost.eps', 'epsc')
f = fopen('output_dir/min_error_rate.tex', 'w'); 
fprintf(f, '%s', min(error_rates(:))); 
min_mat = min(error_rates(:));
[sig_min,cost_min] = find(error_rates==min_mat);
fclose(f);
%{
\end{matlab}

\begin{figure}[h]
\centering
\includegraphics[width=4in]{figures/sigma_cost.eps}
\caption{Error rates for different sigma cost combinations}
\label{fig:sigma_cost} 
\end{figure}

Figure \ref{fig:sigma_cost} illustrates how varying different values of
sigma and cost changes the error rate. The best error rate that was found
using this search is \input{output_dir/min_error_rate}.

To illustrate what these values mean for actually separating the data, we
show a plot of a good sigma, cost pair and a bad one. 

\begin{matlab}
%}
%%
min_mat = min(error_rates(:));
[sig_min,gamma_min] = find(error_rates==min_mat);
gamma_min = gamma_vals(gamma_min);
gamma_min = gamma_min(1); 
sigma_min = sigma_vals(sig_min);
sigma_min = sigma_min(1); 
fprintf('gamma min: %f, sigma_min : %f', gamma_min, sigma_min)
fib_fig = plot_fibonacci_spirals(x1_test, x2_test); 
[plot_3d, contour_plot ] = plot_decision(x_train', y_train, gamma_min, sigma_min, 100, fib_fig);
saveas(contour_plot, 'figures/contour_plot_fib.png')
saveas(plot_3d, 'figures/plot_3d_fib.png')

bad_sig = .0001;
bad_gam = .001;
fib_fig = plot_fibonacci_spirals(x1_test, x2_test);
[plot_3d, contour_plot ] = plot_decision(x_train', y_train, 1, bad_gam, bad_sig, fib_fig);
saveas(contour_plot, 'figures/contour_plot_fib_complex.png')
saveas(plot_3d, 'figures/plot_3d_fib_complex.png')

good_test_error = performance(x_train, y_train, x_train, y_train, gamma_min, sigma_min);
bad_test_error = performance(x_train, y_train, x_train, y_train, bad_gam, bad_sig);

fprintf('Good testing error: %f\n', good_test_error)
fprintf('Bad testing error: %f\n', bad_test_error)

%{
\end{matlab}

Both Figure \ref{fig:sigma_gamma_fib} and Figure \ref{fig:plot_3d_fib} are
the optimal decisions. That is to say, that this is the best trade off
between choosing very small values of sigma and overfitting, and choosing
very large values of sigma and essentially just separating the points with
a single plain. On the other hand, I chose a very low value of sigma in
both Figures \ref{fig:sigma_gamma_fib_overfit} and Figure
\ref{fig:plot_3d_fib_overfit} and it's clear that the hyperplane is seeing
separations in the train data that do not exist in the testing data.  The training error
for Figures \ref{fig:plot_3d_fib} and \ref{fig:sigma_gamma_fib} is:

\color{lightgray}
\VerbatimInput{output_dir/test_train_errors.tex}
\color{black}

This proves that even thought we can have perfect training error, the testing
error can be much worse. This is due to a phenomenon called overfitting. In fact, 
you can choose a sigma and a cost such that you will always be able to get 
100\% training accuracy, but the model will not be good when faced with fitting
the testing data. 

\begin{figure}[h]
\centering
\includegraphics[width=4in]{figures/contour_plot_fib.png}
\caption{Separation of points using optimal sigma and gamma value}
\label{fig:sigma_gamma_fib} 
\end{figure}


\begin{figure}[h]
\centering
\includegraphics[width=4in]{figures/plot_3d_fib.png}
\caption{Separation of points in 3d}
\label{fig:plot_3d_fib} 
\end{figure}

\begin{figure}[h]
\centering
\includegraphics[width=4in]{figures/contour_plot_fib_complex.png}
\caption{Separation of points with overfitting}
\label{fig:sigma_gamma_fib_overfit} 
\end{figure}


\begin{figure}[h]
\centering
\includegraphics[width=4in]{figures/plot_3d_fib_complex.png}
\caption{Separation of points in 3d with overfitting}
\label{fig:plot_3d_fib_overfit} 
\end{figure}

\section{Conclusion}
From these experiments we can see that using kernels is a very helpful in creating a hyperplane
in a higher-dimensional space than the one in which our data is represented. 
Even for complex sets of data such as the Fibonacci Spiral, 
we can still have an accurate separation of classes. These techniques must also be used
with caution because if cross validation is not used, it is guaranteed that you will overfit the data. In this paper we
 demonstrated what will 
happen if only the testing error, or empirical risk is used to create a model using reproducing 
kernels. We find that even though we have driven our empirical risk close to zero, the structural 
risk will begin to to rise as a result. 

Though there are limitations to using kernels in combination with support vector machines, they are very 
good at classifying and regressing data of any distribution which makes them a powerful tool 
for any type of classification.

\bibliography{hw2.bib}
\bibliographystyle{ieeetr}

\end{document}  
%}


